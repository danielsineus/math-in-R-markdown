% Options for packages loaded elsewhere
\PassOptionsToPackage{unicode}{hyperref}
\PassOptionsToPackage{hyphens}{url}
%
\documentclass[
]{article}
\usepackage{lmodern}
\usepackage{amssymb,amsmath}
\usepackage{ifxetex,ifluatex}
\ifnum 0\ifxetex 1\fi\ifluatex 1\fi=0 % if pdftex
  \usepackage[T1]{fontenc}
  \usepackage[utf8]{inputenc}
  \usepackage{textcomp} % provide euro and other symbols
\else % if luatex or xetex
  \usepackage{unicode-math}
  \defaultfontfeatures{Scale=MatchLowercase}
  \defaultfontfeatures[\rmfamily]{Ligatures=TeX,Scale=1}
\fi
% Use upquote if available, for straight quotes in verbatim environments
\IfFileExists{upquote.sty}{\usepackage{upquote}}{}
\IfFileExists{microtype.sty}{% use microtype if available
  \usepackage[]{microtype}
  \UseMicrotypeSet[protrusion]{basicmath} % disable protrusion for tt fonts
}{}
\makeatletter
\@ifundefined{KOMAClassName}{% if non-KOMA class
  \IfFileExists{parskip.sty}{%
    \usepackage{parskip}
  }{% else
    \setlength{\parindent}{0pt}
    \setlength{\parskip}{6pt plus 2pt minus 1pt}}
}{% if KOMA class
  \KOMAoptions{parskip=half}}
\makeatother
\usepackage{xcolor}
\IfFileExists{xurl.sty}{\usepackage{xurl}}{} % add URL line breaks if available
\IfFileExists{bookmark.sty}{\usepackage{bookmark}}{\usepackage{hyperref}}
\hypersetup{
  pdftitle={Bootcamp about Regression},
  pdfauthor={Daniel Sineus},
  hidelinks,
  pdfcreator={LaTeX via pandoc}}
\urlstyle{same} % disable monospaced font for URLs
\usepackage[margin=1in]{geometry}
\usepackage{color}
\usepackage{fancyvrb}
\newcommand{\VerbBar}{|}
\newcommand{\VERB}{\Verb[commandchars=\\\{\}]}
\DefineVerbatimEnvironment{Highlighting}{Verbatim}{commandchars=\\\{\}}
% Add ',fontsize=\small' for more characters per line
\usepackage{framed}
\definecolor{shadecolor}{RGB}{248,248,248}
\newenvironment{Shaded}{\begin{snugshade}}{\end{snugshade}}
\newcommand{\AlertTok}[1]{\textcolor[rgb]{0.94,0.16,0.16}{#1}}
\newcommand{\AnnotationTok}[1]{\textcolor[rgb]{0.56,0.35,0.01}{\textbf{\textit{#1}}}}
\newcommand{\AttributeTok}[1]{\textcolor[rgb]{0.77,0.63,0.00}{#1}}
\newcommand{\BaseNTok}[1]{\textcolor[rgb]{0.00,0.00,0.81}{#1}}
\newcommand{\BuiltInTok}[1]{#1}
\newcommand{\CharTok}[1]{\textcolor[rgb]{0.31,0.60,0.02}{#1}}
\newcommand{\CommentTok}[1]{\textcolor[rgb]{0.56,0.35,0.01}{\textit{#1}}}
\newcommand{\CommentVarTok}[1]{\textcolor[rgb]{0.56,0.35,0.01}{\textbf{\textit{#1}}}}
\newcommand{\ConstantTok}[1]{\textcolor[rgb]{0.00,0.00,0.00}{#1}}
\newcommand{\ControlFlowTok}[1]{\textcolor[rgb]{0.13,0.29,0.53}{\textbf{#1}}}
\newcommand{\DataTypeTok}[1]{\textcolor[rgb]{0.13,0.29,0.53}{#1}}
\newcommand{\DecValTok}[1]{\textcolor[rgb]{0.00,0.00,0.81}{#1}}
\newcommand{\DocumentationTok}[1]{\textcolor[rgb]{0.56,0.35,0.01}{\textbf{\textit{#1}}}}
\newcommand{\ErrorTok}[1]{\textcolor[rgb]{0.64,0.00,0.00}{\textbf{#1}}}
\newcommand{\ExtensionTok}[1]{#1}
\newcommand{\FloatTok}[1]{\textcolor[rgb]{0.00,0.00,0.81}{#1}}
\newcommand{\FunctionTok}[1]{\textcolor[rgb]{0.00,0.00,0.00}{#1}}
\newcommand{\ImportTok}[1]{#1}
\newcommand{\InformationTok}[1]{\textcolor[rgb]{0.56,0.35,0.01}{\textbf{\textit{#1}}}}
\newcommand{\KeywordTok}[1]{\textcolor[rgb]{0.13,0.29,0.53}{\textbf{#1}}}
\newcommand{\NormalTok}[1]{#1}
\newcommand{\OperatorTok}[1]{\textcolor[rgb]{0.81,0.36,0.00}{\textbf{#1}}}
\newcommand{\OtherTok}[1]{\textcolor[rgb]{0.56,0.35,0.01}{#1}}
\newcommand{\PreprocessorTok}[1]{\textcolor[rgb]{0.56,0.35,0.01}{\textit{#1}}}
\newcommand{\RegionMarkerTok}[1]{#1}
\newcommand{\SpecialCharTok}[1]{\textcolor[rgb]{0.00,0.00,0.00}{#1}}
\newcommand{\SpecialStringTok}[1]{\textcolor[rgb]{0.31,0.60,0.02}{#1}}
\newcommand{\StringTok}[1]{\textcolor[rgb]{0.31,0.60,0.02}{#1}}
\newcommand{\VariableTok}[1]{\textcolor[rgb]{0.00,0.00,0.00}{#1}}
\newcommand{\VerbatimStringTok}[1]{\textcolor[rgb]{0.31,0.60,0.02}{#1}}
\newcommand{\WarningTok}[1]{\textcolor[rgb]{0.56,0.35,0.01}{\textbf{\textit{#1}}}}
\usepackage{graphicx,grffile}
\makeatletter
\def\maxwidth{\ifdim\Gin@nat@width>\linewidth\linewidth\else\Gin@nat@width\fi}
\def\maxheight{\ifdim\Gin@nat@height>\textheight\textheight\else\Gin@nat@height\fi}
\makeatother
% Scale images if necessary, so that they will not overflow the page
% margins by default, and it is still possible to overwrite the defaults
% using explicit options in \includegraphics[width, height, ...]{}
\setkeys{Gin}{width=\maxwidth,height=\maxheight,keepaspectratio}
% Set default figure placement to htbp
\makeatletter
\def\fps@figure{htbp}
\makeatother
\setlength{\emergencystretch}{3em} % prevent overfull lines
\providecommand{\tightlist}{%
  \setlength{\itemsep}{0pt}\setlength{\parskip}{0pt}}
\setcounter{secnumdepth}{-\maxdimen} % remove section numbering

\title{Bootcamp about Regression}
\author{Daniel Sineus}
\date{11/10/2020}

\begin{document}
\maketitle

{
\setcounter{tocdepth}{4}
\tableofcontents
}
\newpage

This is a document written in the objective of answering certain
questions in relation to a linear regression model, a forecasting model,
general questions about equation and at last monetary policy of FED. The
document is subdivided in four sections. Each section is attempting to
elucidate my answers to all the questions, except the third section in
which I will only present the solutions to the equations. As to the
solution of the equations, they will be presented on a fraction format.
This paper has been written in a Markdown format, because I want to use
LaTex so the equation can be readable and look attractive to the eyes.
It is simply a personal choice. Markdown is a simple formatting syntax
for authoring HTML, PDF, and MS Word documents.

\hypertarget{question-1---analysis-of-the-linear-regression-model}{%
\section{Question 1 - Analysis of the Linear Regression
Model}\label{question-1---analysis-of-the-linear-regression-model}}

In fact before going further, in a general way, one of the purposes of
having a model is to predict the outcome of a situation when a
independent variable changes. They can allow people to simulate
scenarios. We will analyze the predictors underscored here in order to
determine if the predictors are significant enough to predict the family
income. Income of house can be a function of these independent
variables. \(f(income)=({value,education, mortgage, age, gender})\), the
best way to outline the function:
\(family-income =28.37 + 28*Value + 0.652*Education - 0.050*Age - 0.001*Mortgage + 0.751*Gender)\).
The purpose in the homework is not to interpret the model, but I can
give some examples to interpret the model. I would simulate the model
randomly by attributing the variables. For each additional value of home
of \($1000\), the family income will increase by \($28\) on an average.
Family income will increase on average by \($.652\) for each additional
year getting education. An increase of a year of age after the means of
age, \(37\), will decrease the family income on average by \($.050\). On
average, the family income will increase \(0.751\)times greater when the
head of the household is male than when it is a female. On average, the
family income will decrease by \($ 0.001\) when the monthly mortgage
payment increases by a one dollar.

\hypertarget{effectiveness-of-the-predictors-of-the-model}{%
\subsection{Effectiveness of the predictors of the
model}\label{effectiveness-of-the-predictors-of-the-model}}

Given the fact that F-statistic/test shows a P-value less than 0.05, it
is conclusive to accept that the overall association of all independent
variables in the regression model, otherwise it could have been
rejected. The overall significance of the model is based on the F-test.
The model is effective in this particular situation. But when we
consider the T-test which presents the significance of individual
variables, the predictors value of the house, years of education and
gender are significant with their respective P-value less than 0.05. It
implies that these variables are effective predictors individually,
however the regressors age and monthly mortgage payment are not
significant to predict the family income. They have a P-value higher
than 0.05. Therefore, they are not effective predictors to the model. It
is a good thing that the R-square is 74.8\%. It means that 74.8\% of the
variation of the family income is related to the variation of the
regressors such as value, education, gender, more or less mortgage
payment and age.

\hypertarget{recommendation-to-improve-the-predictability-of-the-model}{%
\subsection{Recommendation to improve the predictability of the
Model}\label{recommendation-to-improve-the-predictability-of-the-model}}

First of all, for such a regression model, there needs to be a larger
sample size than an observation of 25. The formulation of the model has
not shown a priori any sign of multicolinearity, which is the occurrence
of high intercorrelation among two or more independent variables (Rhodd,
2020). This assumption has not been proved by a statistical test here
since it is not the objective but it is still valid when considering the
variables.

I strongly recommend that the variable ``income tax declared by the
households'' be incorporated in the model to predict family income. Some
explicit information has not be given about the regressand ``family
income''. Is it the objective of model to determine the family income on
a yearly, quarterly, monthly basis? This specificity that is left out
makes our interpretation less accurate and precise.

I made the assumption that the bank needs the model for giving loans or
financing projects for its customers. In regard to this assumption, the
bank fails to conduct the appropriate model for its benefit in terms of
a bigger picture. Why would the bank try to determine the income of a
family whereas it can ask the households to submit all documents that
can weigh in their favor? I would recommend the bank to conduct a
logistic regression that can predict loan defaults.Thus, the bank can
know in advance the customer who will be likely to default.

Some explicit information has not been given about the depend variable.
This lack of information makes difficult to interpret the dependent
variable with the precision but not the regression model. the question
we ask ourselves, is the income of households that the bank is trying to
determine yearly, quartely or monthly? I would incorporate income tax as
a valuable independent variable. But we have to agree that it would
increase the \(r^2\), the r-squared.According to Woolridge (2016), ``an
important fact about \(r^2\) is that it never decreases. it usually
increases when independent variables variable is added to a
regression''. That is to say, by incorporating the income tax as a
regressand, it will increase the r squared. Instead of having R-square
of 0.748, the regression statistic will display a higher one.

\hypertarget{question-2---choice-of-the-best-forecasting-model}{%
\section{Question 2 - Choice of the Best Forecasting
Model}\label{question-2---choice-of-the-best-forecasting-model}}

\hypertarget{growth-rate-and-average-growth-rate}{%
\subsection{Growth Rate and Average Growth
Rate}\label{growth-rate-and-average-growth-rate}}

The monthly growth rate will be presented in a table. Since it is an
academic homework, I find it appropriate to present the steps even
though it doesn't really matter. It is done in case someone would like
to know how I come down to the results

\begin{Shaded}
\begin{Highlighting}[]
\CommentTok{##Since it is a homework, I let the codes appear.}

\KeywordTok{library}\NormalTok{(dplyr)}
\end{Highlighting}
\end{Shaded}

\begin{verbatim}
TRUE 
TRUE Attaching package: 'dplyr'
\end{verbatim}

\begin{verbatim}
TRUE The following objects are masked from 'package:stats':
TRUE 
TRUE     filter, lag
\end{verbatim}

\begin{verbatim}
TRUE The following objects are masked from 'package:base':
TRUE 
TRUE     intersect, setdiff, setequal, union
\end{verbatim}

\begin{Shaded}
\begin{Highlighting}[]
\KeywordTok{library}\NormalTok{(tidyr)}
\NormalTok{Period=}\KeywordTok{c}\NormalTok{(}\StringTok{"Month 1"}\NormalTok{, }\StringTok{"Month 2"}\NormalTok{, }\StringTok{"Month 3"}\NormalTok{, }\StringTok{"Month 4"}\NormalTok{, }
         \StringTok{"Month 5"}\NormalTok{, }\StringTok{"Month 6"}\NormalTok{, }\StringTok{"Month 7"}\NormalTok{, }\StringTok{"Month 8"}\NormalTok{, }
         \StringTok{"Month 9"}\NormalTok{, }\StringTok{"Month 10"}\NormalTok{, }\StringTok{"Month 11"}\NormalTok{, }\StringTok{"Month 12"}\NormalTok{)}
\NormalTok{Actual_Value =}\StringTok{ }\KeywordTok{c}\NormalTok{(}\DecValTok{10}\NormalTok{, }\DecValTok{12}\NormalTok{, }\DecValTok{16}\NormalTok{, }\DecValTok{13}\NormalTok{, }\DecValTok{17}\NormalTok{, }\DecValTok{19}\NormalTok{, }\DecValTok{15}\NormalTok{,  }\DecValTok{20}\NormalTok{, }\DecValTok{22}\NormalTok{, }\DecValTok{19}\NormalTok{, }\DecValTok{21}\NormalTok{, }\DecValTok{19}\NormalTok{)}
\NormalTok{data=}\KeywordTok{data.frame}\NormalTok{(Period,Actual_Value)}
\NormalTok{growth_data=data}\OperatorTok
\StringTok{  }\KeywordTok{mutate}\NormalTok{(}\DataTypeTok{Growth_rate=}\NormalTok{(Actual_Value}\OperatorTok{-}\KeywordTok{lag}\NormalTok{(Actual_Value))}\OperatorTok{/}\KeywordTok{lag}\NormalTok{(Actual_Value)}\OperatorTok{*}\DecValTok{100}\NormalTok{,}
         \DataTypeTok{rounded=}\KeywordTok{round}\NormalTok{(Growth_rate,}\DecValTok{2}\NormalTok{))}
\NormalTok{average_growth=}\KeywordTok{mean}\NormalTok{(growth_data}\OperatorTok{$}\NormalTok{rounded, }\DataTypeTok{na.rm =} \OtherTok{TRUE}\NormalTok{)}
\end{Highlighting}
\end{Shaded}

\begin{Shaded}
\begin{Highlighting}[]
\NormalTok{growth_data}
\end{Highlighting}
\end{Shaded}

\begin{verbatim}
##      Period Actual_Value Growth_rate rounded
## 1   Month 1           10          NA      NA
## 2   Month 2           12    20.00000   20.00
## 3   Month 3           16    33.33333   33.33
## 4   Month 4           13   -18.75000  -18.75
## 5   Month 5           17    30.76923   30.77
## 6   Month 6           19    11.76471   11.76
## 7   Month 7           15   -21.05263  -21.05
## 8   Month 8           20    33.33333   33.33
## 9   Month 9           22    10.00000   10.00
## 10 Month 10           19   -13.63636  -13.64
## 11 Month 11           21    10.52632   10.53
## 12 Month 12           19    -9.52381   -9.52
\end{verbatim}

\textbf{The average month growth rate is \(7.8872\)}.

\begin{verbatim}
## [1] 7.887273
\end{verbatim}

\hypertarget{the-best-fit-model}{%
\subsection{The best fit Model}\label{the-best-fit-model}}

To decide the Model that is the best fit, we have to look at the Mean
Absolute Percent Error (MAPE) and choose the smaller MAPE. In this case,
based on the three techniques used, We would choose the \textbf{three
period moving average} to predict the next period's growth rate because
it presents the lowest MAPE or it has the smallest deviation.

\hypertarget{solving-the-following-equations}{%
\section{Solving the following
equations}\label{solving-the-following-equations}}

\begin{enumerate}
\def\labelenumi{\alph{enumi}.}
\item
\end{enumerate}

Value after 8 years by using a growth factor

\(22500\times(1-\frac{1}{100})^8=9685.5122\)

\begin{enumerate}
\def\labelenumi{\alph{enumi}.}
\setcounter{enumi}{1}
\item
\end{enumerate}

Number of Units needed to the firm to breakeven

\begin{align*} 
Profit =total revenue-total cost\\
total cost = variable cost + fixed cost\\
$60000=12q-(9q+90000)\\
60000=12q-9q-90000\\
60000+90000=3q\\
q=\frac{(60000+90000)}{3}\\
q=50000
\end{align*}

\begin{enumerate}
\def\labelenumi{\alph{enumi}.}
\setcounter{enumi}{2}
\item
\end{enumerate}

\begin{align*}-6(3x-2)=16(x-4)\Longrightarrow\ &-3(3x-2)=8(x-4)\\
&-9x-8x=-4-6\\
&-17x=-10\\
&x=\frac{10}{17}
\end{align*}

\begin{enumerate}
\def\labelenumi{\alph{enumi}.}
\setcounter{enumi}{3}
\item
\end{enumerate}

\begin{align*}\frac{6}{2x+9}=\frac{2}{x+2}\longrightarrow\ 6(x+2)=2(2x+9)\\
& 6x+12=4x+18\\
& 6x-4x=18-12\\
& 2x=6 \Longrightarrow x=3
\end{align*}

\begin{enumerate}
\def\labelenumi{\alph{enumi}.}
\setcounter{enumi}{4}
\item
\end{enumerate}

\begin{cases} x-3y=-25 \\ 4x+5y=19
\end{cases}

\textbf{multiplied by -4 the first equation}

\begin{cases} -4x +12y=100 \\ 4x+5y =19
\end{cases}

subtract the equations

\begin{align*}
(-4x+4x)+(12y+5y)&=100+19\\
&=17y=110\\
&y= \frac{110}{17}\\
\end{align*}

\begin{enumerate}
\def\labelenumi{\alph{enumi}.}
\setcounter{enumi}{5}
\item
\end{enumerate}

\begin{align*} \frac{(3^-3\times 4^5)}{4^2\times 5^2} &=\frac{(\frac{1}{3^3} \times 4^5)}{4^2 \times 5^2}\\
&=\frac{\frac{4^5}{3^3}} {4^2\times 5^2}\\
&=\frac{4^5}{27}\times \frac{1}{4^2 \times 5^2}\\
&=\frac{4^{5-2}}{27\times5^2}\\
&=\frac{4^3}{27\times 5 \times 5}\\
&=\frac{64}{675}
\end{align*}

\hypertarget{question---understanding-some-concepts-about-federal-reserve}{%
\section{Question - Understanding some concepts about Federal
reserve}\label{question---understanding-some-concepts-about-federal-reserve}}

This section will attempt to explain the reasons why the Federal Reserve
used open market operations,the discount rate, and the reserve
requirement to reduce inflation. Afterward, I will present shortly the
three policies and in the way mention the reasons among the three why a
policy matters most and best than the two others.

\hypertarget{how-the-fed-reduces-inflation}{%
\subsection{How the FED reduces
inflation}\label{how-the-fed-reduces-inflation}}

Before going straight to the question, it is important that each of the
economic goals of the FED is well defined. Once we understand each
economic goals of FED, subsequently, we will be able to relate it to how
the FED reduces inflation. The goals of monetary policy are to promote
maximum employment, stable prices and moderate long-term interest rates.
The \textbf{discount rate} is the interest rate Reserve Banks charge
commercial banks for short-term loans. Federal Reserve lending at the
discount rate complements open market operations in achieving the target
federal funds rate and serves as a backup source of liquidity for
commercial banks. Lowering the discount rate is expansionary because the
discount rate influences other interest rates. Lower rates encourage
lending and spending by consumers and businesses. Likewise, raising the
discount rate is contractionary because the discount rate influences
other interest rates. Higher rates discourage lending and spending by
consumers and businesses. \textbf{Reserve requirements} are the portions
of deposits that banks must hold in cash, either in their vaults or on
deposit at a Reserve Bank. A decrease in reserve requirements is
expansionary because it increases the funds available in the banking
system to lend to consumers and businesses. An increase in reserve
requirements is contractionary because it reduces the funds available in
the banking system to lend to consumers and businesses. \textbf{Open
market operations} are when central banks buy or sell securities. These
are bought from or sold to the country's private banks. When the central
bank buys securities, it adds cash to the banks' reserves. That gives
them more money to lend. When the central bank sells the securities, it
places them on the banks' balance sheets and reduces its cash holdings.

To get to the bottom of the question, The Federal Reserve works to
promote a strong U.S. economy. The FED's goals of maximum employment,
stable prices, and moderate long-term interest rates. When prices are
stable, long-term interest rates remain at moderate levels, so the goals
of price stability and moderate long-term interest rates go together. As
a result, it influences employment and inflation primarily through using
its policy tools to influence the availability and cost of credit in the
economy. When interest rates go down, it becomes cheaper to borrow, so
households are more willing to buy goods and services, and businesses
are in a better position to purchase items to expand their businesses,
such as property and equipment. Businesses can also hire more workers,
influencing employment. And the stronger demand for goods and services
may push wages and other costs higher. Thus it will influence inflation.

\textbf{The conclusion that we come to, as interest rates are reduced,
more people are able to borrow more money. The result is that consumers
have more money to spend. This causes the economy to grow and inflation
to increase. By the same token, When the FED wants to reduce the
inflation, the FED increases interest rate}.

\hypertarget{the-most-effective-policy-of-all-three-policies}{%
\subsection{The most effective policy of all three
policies}\label{the-most-effective-policy-of-all-three-policies}}

From my standpoint and in the light of everything aformentioned, the
FED's policy related to interest rate is the most effective one. because
it is the primary tool to influence employment and inflation.

\hypertarget{bibliography}{%
\section{Bibliography}\label{bibliography}}

Federal Reserve.(2020).Federal Reserve Board - Monetary Policy.
Retrieved from \url{https://www.federalreserve.gov/monetarypolicy.htm}.

Federal Reserve Edication.org. (n.d). Monetary Policy Basics. Retrieved
from:
\url{https://www.federalreserveeducation.org/about-the-fed/structure-and-functions/monetary-policy}.

Rhodd.(2020). Correlation Regression Analysis and Forecasting. Florida
Atlantic University.Retrieved from:
\url{https://canvas.fau.edu/courses/92000/files/20268406?module_item_id=2296074}.

\end{document}
